\problemname{Evening Fika}
\noindent
Sara loves fika! She has just received $P$ Swedish kronor from her mom to buy some evening fika.
Since Sara is smart, she will try to spend the money as wisely as possible. At the store Lugnbyrån,
there are $N$ pastries available. Each pastry has a certain price and a specific category.
Because Sara likes fika so much, she \textit{almost} thinks all kinds of pastries are equally tasty.
Namely, she has no preference for which category of pastry to buy, but she feels it would be
a bit too monotonous if she buys more than $K$ pastries of the same category.

Help Sara buy as many pastries as possible, with a total cost less than or equal to $P$.
Also, ensure that there are no more than $K$ pastries from the same category among those she bought.
You only need to print out how many pastries she can buy, not which ones.

\section*{Input}
The first line of the input contains the integer $N$ ($1 \leq N \leq 10^5$), the number of pastries
Sara can choose from.

The next line contains the integer $P$ ($1 \leq P \leq 10^8$), the number of kronor Sara has as a
budget for her evening fika.

The next line contains the integer $K$ ($1 \leq K \leq N$), the maximum number of pastries from
each category that Sara is willing to eat.

The next line contains the integers $c_1, c_2, \dots, c_N$ ($1 \leq c_i \leq P$), where $c_i$ means that
that pastry $i$ costs $c_i$ kronor. Note that the total cost of all pastries may not necessarily fit
in a 32-bit integer.

The last line contains the integers $t_1, t_2, \dots, t_N$ ($1 \leq t_i \leq 10^5$). There are
exactly $10^5$ categories of pastries, and we refer to each with an integer. Each $t_i$ means that
that pastry $i$ belongs to category $t_i$.


\section*{Output}
Print an integer: the number of pastries that Sara can buy if she uses her budget optimally.

\section*{Scoring}
Your solution will be tested on a set of test groups, each worth a number of points. Each test group contains
a set of test cases. To get the points for a test group you need to solve all test cases in the test group.

\noindent
\begin{tabular}{| l | l | p{12cm} |}
  \hline
  \textbf{Group} & \textbf{Points} & \textbf{Constraints} \\ \hline
  $1$    & $20$       & The subtask has a single testcase, the one on the poster (\url{https://www.progolymp.se/2025/affisch.pdf}). \\ \hline 
  $2$    & $5$        & $K=N$ and you can afford to buy every single pastry. \\ \hline
  $3$    & $15$       & You can afford to buy every single pastry. \\ \hline
  $4$    & $5$        & $c_i=1$, that is to say every pastry costs $1$ krona. \\ \hline
  $5$    & $17$       & $K=N, N \leq 500$ \\ \hline
  $6$    & $21$       & $K=N$ \\ \hline
  $7$    & $9$        & $N \leq 500$ \\ \hline
  $8$    & $8$        & No additional constraints. \\ \hline
\end{tabular}

\section*{Explanation of Sample 1}
In this sample, Sara can afford to buy all the pastries, but they all belong to the same category (category 1).
Since she can buy at most 2 per category, she can purchase 2 of them.

\section*{Explanation of Sample 2}
In this sample, categories are not a problem because $K$ is large, but Sara cannot afford to buy all the pastries.
If she buys the pastries priced at $4, 3, 2, 1$, she can afford exactly 4.

\section*{Explanation of Sample 3}
An optimal solution is to buy pastries priced at $1, 1, 4, 5$. Here, Sara could have bought more pastries if not
limited by the categories.
