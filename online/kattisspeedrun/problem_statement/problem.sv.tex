\problemname{Kattis Speedrun}
Joshuas nyårslöfte för 2024 var att lösa hälften av alla problem på hemsidan Kattis innan året är slut. 
Nu börjar året närma sig sitt slut och han börjar inse att han kommer att behöva använda allt han kan för att nå sitt mål.
Hittills har har bara löst andras problem på Kattis. Men han kan också skapa ett eget problem, ladda upp det och lösa det.
Detta innebär ibland mindre arbete än att lösa någon annans problem, men leder också till att totala antalet problem ökar.

Idag finns det $P$ stycken problem på Kattis, och Joshua har löst $S$ av dessa. Han kan göra följande saker:

\begin{itemize}
  \item Lösa ett problem. Detta tar $A$ sekunder och ökar $S$ med 1.
  \item Skriva ett nytt problem. Detta tar $B$ sekunder och ökar både $P$ och $S$ med 1.
\end{itemize}

Om Joshua använder sin tid optimalt, hur många sekunder kommer det ta för honom att öka $S$ och $P$ så att
$\frac{S}{P} > 0.5$?

\section*{Indata}
Den första raden i indatan innehåller heltalet $S$ ($0 \leq S \leq 10^5$), antalet problem som Joshua har
löst hittills.

Den andra raden innehåller heltalet $P$ ($1 \leq P \leq 10^5$), totala antalet problem som finns
på Kattis hittills. Det är garanterat att $S < \frac{P}{2}$.

Den tredje raden innehåller heltalet $A$ ($1 \leq A \leq 10^5$), tiden det tar att lösa ett problem.

Den fjärde raden innehåller heltalet $B$ ($1 \leq B \leq 10^5$), tiden det tar att skapa ett nytt problem.

Notera att i vissa testfall kan $A$ vara väldigt mycket större än $B$, och i andra testfall kan $B$ vara mycket större.

\section*{Utdata}
Skriv ut ett heltal: minsta antalet sekunder det kommer ta Joshua för att lösa strikt mer än hälften
av alla problem på Kattis.

\section*{Poängsättning}
Din lösning kommer att testas på en mängd testfallsgrupper.
För att få poäng för en grupp så måste du klara alla testfall i gruppen.

\noindent
\begin{tabular}{| l | l | p{12cm} |}
  \hline
  \textbf{Grupp} & \textbf{Poäng} & \textbf{Gränser} \\ \hline
  $1$    & $12$       & $A=1, B=10^5$ \\ \hline
  $2$    & $18$       & $B=1, A=10^5$ \\ \hline
  $3$    & $20$       & $S, P \leq 1000$ \\ \hline
  $4$    & $50$       & Inga ytterligare begränsningar. \\ \hline
\end{tabular}

\section*{Förklaring av exempelfall}
I exempelfall 1 går det jättesnabbt att lösa problem, men tar jättelång tid att skapa nya. Optimala strategin
är därmed att bara lösa problem. För att lösa mer än hälften måste han lösa $51$ stycken. Eftersom $A=1$ blir svaret
$51$.

I exempelfall 2 måste 2 problem lösas för att lösa mer än hälften.

I exempelfall 3 är den optimala strategin att lösa 11 problem och skapa 1 problem. Detta tar sammanlagt $11 \cdot 7 + 1 \cdot 6=83$ sekunder.
Det finns ingen annan strategi som är snabbare.
