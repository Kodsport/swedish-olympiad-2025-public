\problemname{Siffrids siffersumma}
\noindent
Siffrid älskar att leka med tal! Just nu sitter hon och leker med heltalet $N$.

Hon undrar hur man kan skapa det minsta och största talet som har samma siffersumma och antal siffror som $N$.
Kan du hjälpa henne?

Siffersumman av ett tal definieras som summan av alla siffror i talet. 

\noindent
Till exempel är siffersumman av talet $1234$: 
$$
1 + 2 + 3 + 4 = 10,
$$

och siffersumman av talet $220$:
$$
2 + 2 + 0 = 4.
$$


\section*{Indata}
Indatan består av en rad med ett heltal, $N$ ($1 \le N \le 10^9$).

\section*{Utdata}
Skriv ut två heltal på samma rad, det minsta och det största talet med samma siffersumma och antal siffror som $N$.
Talen ska vara separerade med ett mellanslag.

\section*{Poängsättning}
Din lösning kommer att testas på en mängd testfallsgrupper.
För att få poäng för en grupp så måste du klara alla testfall i gruppen.

\noindent
\begin{tabular}{| l | l | p{12cm} |}
  \hline
  \textbf{Grupp} & \textbf{Poäng} & \textbf{Gränser} \\ \hline
  $1$    & $40$       & $N \leq 10^5$ \\ \hline
  $2$    & $60$       & Inga ytterligare begränsningar. \\ \hline
\end{tabular}

\section*{Förklaring av exempelfall 1}
Vi kan inte få ett tal mindre än 101. Tal som 11 och 2 har samma siffersumma som $N$ och är mindre än 101, men de har inte
lika många siffror som $N$.
