\problemname{Siffrid's Digit Sum}

Siffrid loves playing with numbers! Right now, she is playing with the integer $N$.

She wonders how to create the smallest and largest number that have the same digit sum and number of digits as $N$.
Can you help her?

The digit sum of a number is defined as the sum of all its digits.

\noindent
For example, the digit sum of the number $1234$ is:
$$
1 + 2 + 3 + 4 = 10,  
$$

and the digit sum of the number $220$ is:
$$
2 + 2 + 0 = 4.
$$

\section*{Input}
Input consists of the single integer $N$ ($1 \le N \le 10^9$).

\section*{Output}
Print two integers on the same line: the smallest and largest number with the same digit sum and
number of digits as $N$. The numbers should be separated by a space.

\section*{Scoring}
Your solution will be tested on a set of test groups, each worth a number of points. Each test group contains
a set of test cases. To get the points for a test group you need to solve all test cases in the test group.

\noindent
\begin{tabular}{| l | l | p{12cm} |}
  \hline
  \textbf{Group} & \textbf{Points} & \textbf{Constraints} \\ \hline
  $1$    & $40$       & $N \leq 10^5$ \\ \hline
  $2$    & $60$       & No additional constraints. \\ \hline
\end{tabular}

\section*{Explanation of Sample 1}
We cannot create a number smaller than $101$ with the same digit sum and number of digits as $N$.
Numbers like $11$ and $2$ have the same digit sum as $N$ but do not have the same number of digits as $101$.
