\problemname{Virus}

\noindent

Boschua är inte så bra på cybersäkerhet. Förra månaden stängde han av brandväggen för att den
inte lät honom ladda ner kattbilder. Tyvärr har detta lett till att han fått ett virus. 
Viruset gjorde något väldigt konstigt: det lade till en massa bokstäver i alla filnamn.

Exempelvis kan det ha ändrat filnamnet \texttt{katt.png} till \texttt{k\textbf{atteg}att\textbf{o}.p\textbf{i}ng}.
Viruset kan alltså lägga till bokstäver, men aldrig ta bort bokstäver.

Han letar nu  efter filen som heter $F$, men han har jättemånga filer. Kan du skriva ett program som läser in filnamnet $F$ och namnet på en fil $H$
på hans dator, och avgör om $F$ kan ha gjorts om till $H$ av viruset? Det vill säga, är det möjligt att
viruset tog filnamnet $F$ och lade till bokstäver så att filen nu heter $H$?


\section*{Indata}
Den första raden av indata innehåller strängen $F$ och den andra raden innehåller strängen.

Låt $|F|$ vara antalet bokstäver i $F$, och detsamma för $|H|$.
Då håller det att $1 \le |F| \le |H| \le 32$. $F$ och $H$ innehåller endast bokstäver \texttt{a}-\texttt{z} och punkt.

Det är garanterat att $F$ innehåller exakt en punkt. Viruset kan lägga till punkter och vilken bokstav från \texttt{a}-\texttt{z} som helst.


\section*{Utdata}
Skriv ut ''Ja'' om $H$ kan vara filen Boschua letar efter, annars ''Nej''.

\section*{Poängsättning}
Din lösning kommer att testas på en mängd testfallsgrupper.
För att få poäng för en grupp så måste du klara alla testfall i gruppen.


\noindent
\begin{tabular}{| l | l | p{12cm} |}
  \hline
  \textbf{Grupp} & \textbf{Poäng} & \textbf{Gränser} \\ \hline
  $1$    & $20$       & Varje bokstav (och punkt) dyker upp som mest en gång i vardera sträng. \\ \hline
  $2$    & $20$       & $|H| = |F|+1$ \\ \hline
  $3$    & $20$       & $|H| = |F|+2$ \\ \hline
  $4$    & $40$       & Inga ytterligare begränsningar. \\ \hline
\end{tabular}


\section*{Förklaring av exempelfall}
I exempelfall 1 kan viruset ha gjort följande förändring för att göra om $F$ till $H$:
\texttt{katt.png} $\Rightarrow$ \texttt{ka\textbf{ttega}tt\textbf{o}.p\textbf{i}ng}.


I exempelfall 2 kan viruset ha gjort följande förändring för att göra om $F$ till $H$:
\texttt{antivirus.exe} $\Rightarrow$ \texttt{a\textbf{aa}ntiv\textbf{is}i\textbf{i}ru\textbf{i}s.e\textbf{g}xe}.


I exempelfall 3 skiljer sig $F$ och $H$ vid \texttt{png} och \texttt{pgn}. Eftersom viruset inte kan ändra ordningen på bokstäver
kan det inte ha ändrat $F$ till $H$. 
