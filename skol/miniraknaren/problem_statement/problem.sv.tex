\problemname{Miniräknaren}
Det går rykten om att skolans miniräknare kan trimmas med en hemlig kod.
Du har hört att om du kan få miniräknaren att visa den hemliga kålnami-koden på displayen och sedan
trycker på likhetstecknet, så kommer miniräknaren att låsa upp flera
stycken hemliga matematikfunktioner.

Din kompis har lyckats lista ut vad kålnami-koden är, men hen berättar för dig
att du inte får gå tillväga hur som helst för att skriva in koden i miniräknaren.
Från början visar miniräknaren $0$. Sedan får du utföra följande operationer:
\begin{itemize}
  \item Multiplicera talet som visas med talet $M$.
  \item Addera ett valfritt tal $k$ till talet som visas, där $k$ uppfyller $0 \leq k \leq M-1$.
\end{itemize}

Som tur är har din kompis även lyckats lista ut vad talet $M$ är. 
Denna komplicerade process kommer endast fungera om du gör så få operationer
som möjligt. Därför vill du nu ta reda på hur många operationer som krävs,
givet kålnami-koden och talet $M$.

\section*{Indata}
Den första raden innehåller heltalet $N$ ($1 \le N \le 10^9$), innehållet av kålnami-koden.

Nästa rad innehåller siffran $M$ ($2 \leq M \leq 9$), som beskrivits ovan.

\section*{Utdata}
Skriv ut ett heltal: minsta antalet operationer som krävs för att få miniräknaren att visa
kålnami-koden $N$.


\section*{Poängsättning}
Din lösning kommer att testas på en mängd testfallsgrupper.
För att få poäng för en grupp så måste du klara alla testfall i gruppen.

\noindent
\begin{tabular}{| l | l | p{12cm} |}
  \hline
  \textbf{Grupp} & \textbf{Poäng} & \textbf{Gränser} \\ \hline
  $1$    & $20$       & $M=3, N \leq 10$. \\ \hline
  $2$    & $20$       & $M=2$ \\ \hline
  $3$    & $20$       & $N \leq 10^5$ \\ \hline
  $4$    & $40$       & Inga ytterligare begränsningar. \\ \hline
\end{tabular}

\section*{Förklaring av exempelfall}
I exempelfall 1 är ett sätt att skriva in koden med minsta antalet operationer som följande:
\begin{enumerate}
  \item $+1$
  \item $\times 2$
  \item $\times 2$
\end{enumerate}


I exempelfall 2 är en optimal lösning är att utföra följande operationer:
\begin{enumerate}
  \item $+1$
  \item $\times 3$
  \item $\times 3$
  \item $+2$
  \item $\times 3$
\end{enumerate}
