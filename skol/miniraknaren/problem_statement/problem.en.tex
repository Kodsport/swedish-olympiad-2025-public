\problemname{The Calculator}

Rumors are circulating that your school-provided calculator can be upgraded using a secret code.  
You've heard that if you can get the calculator to display the secret \textit{kålnami}
code on its screen and then press the equals sign, the calculator will unlock several
secret mathematical functions.  

Your friend has figured out what the kålnami code is, but they tell you that
you can't input the code into the calculator in just any way.  
Initially, the calculator displays $0$. You can then perform the following operations:  
\begin{itemize}  
  \item Multiply the number displayed by $M$.  
  \item Add any number $k$ to the displayed number, where $k$ satisfies $0 \leq k \leq M-1$.  
\end{itemize}  

Fortunately, your friend has also discovered the value of $M$.  
This complicated process will only work if you perform as few operations as possible.
Therefore, you want to determine the minimum number of operations required, given the kålnami
code and the value of $M$.  

\section*{Input}  
The first line of input contains the integer $N$ ($1 \leq N \leq 10^9$), the value of the kålnami code.

The next line contains the digit $M$ ($2 \leq M \leq 9$), whose meaning is described above.

\section*{Output}  
Print a single integer: the minimum number of operations required to make the calculator display the kålnami code $N$.

\section*{Scoring}
Your solution will be tested on a set of test groups, each worth a number of points. Each test group contains
a set of test cases. To get the points for a test group you need to solve all test cases in the test group.

\noindent
\begin{tabular}{| l | l | p{12cm} |}
  \hline
  \textbf{Group} & \textbf{Points} & \textbf{Constraints} \\ \hline
  $1$    & $20$       & $M=3, N \leq 10$. \\ \hline
  $2$    & $20$       & $M=2$ \\ \hline
  $3$    & $20$       & $N \leq 10^5$ \\ \hline
  $4$    & $40$       & No additional constraints. \\ \hline
\end{tabular}

\section*{Explanation of Samples}
In sample 1, one way to input the code with the minimum number of operations is as follows:
\begin{enumerate}
  \item $+1$
  \item $\times 2$
  \item $\times 2$
\end{enumerate}


In sample 2, an optimal solution is to perform the following operations:
\begin{enumerate}
  \item $+1$
  \item $\times 3$
  \item $\times 3$
  \item $+2$
  \item $\times 3$
\end{enumerate}
