\problemname{Truls trubbel}
\noindent

Truls och Henry spelar pingis mot varandra.

Både Truls och Henry ogillar starkt matematik.
Mer än allting annat ogillar de att räkna på antalet poäng som vardera person har under tävlingen.
För att förenkla poängräkningen har de kommit på en egen lösning:

Istället för att ständigt hålla koll på hur många poäng de har, skriver de 
ett ''\texttt{T}'' om Truls får poäng och ett ''\texttt{H}'' om Henry
får poäng, direkt efter varje vunnen boll.

När Truls och Henry spelar pingis fungerar poängsystemet enligt dessa tre regler:
\begin{enumerate}
  \item Vid varje boll delas ett poäng ut.
  \item En spelare vinner en match om båda av de två följande villkoren är uppfyllda:
  \begin{itemize}
    \item Spelaren leder med åtminstone 2 poäng över sin motståndare.
    \item Spelaren har minst 11 poäng.
  \end{itemize}
  Till exempel, om Truls har 11 poäng och Henry har 9 poäng, så utses Truls till vinnare. Men om Truls har 11 poäng och Henry har 10 poäng finns det ännu ingen vinnare.

  \item När en spelare har vunnit enligt ovanstående kriterier, så nollställs poängen för båda spelarna och en ny match startar.
\end{enumerate}

Nu har Truls och Henry spelat väldigt länge, och båda vill veta hur poängen ser ut.
De ber de dig att skriva ett program som
räknar ut den aktuella poängställningen åt dem, baserat på deras anteckningar.


\section*{Indata}
Indatan består av poängsträngen, som består av flera bokstäver på en rad. Antalet bokstäver är mellan 1 och 30. 
Varje bokstav kommer antingen vara ''\texttt{T}'' eller ''\texttt{H}''.

\section*{Utdata}
Skriv ut $T$-$H$, där $T$ är antalet poäng som Truls har, och $H$ är antalet poäng som Henry har.


\section*{Poängsättning}
Din lösning kommer att testas på en mängd testfallsgrupper.
För att få poäng för en grupp så måste du klara alla testfall i gruppen.

\noindent
\begin{tabular}{| l | l | p{12cm} |}
  \hline
  \textbf{Grupp} & \textbf{Poäng} & \textbf{Gränser} \\ \hline
  $1$    & $40$       & En match avgörs alltid så fort någon får 11 poäng. \\ \hline
  $2$    & $60$       & Inga ytterligare begränsningar. \\ \hline
\end{tabular}


\section*{Förklaring av exempelfall 1}
Sammanlagt har Truls vunnit fyra bollar, och Henry 5 bollar.

\section*{Förklaring av exempelfall 2}
Eftersom Truls vann matchen när poängställningen blev \texttt{11-0}, så nollställdes poängen till \texttt{0-0}.

\section*{Förklaring av exempelfall 3}
Även om båda har fler än 11 poäng leder ingen med 2 poäng. Därför har ingen vinnare utsetts ännu.

\section*{Förklaring av exempelfall 4}
Detta spel är en fortsättning på exempelfall 3 där Henry vinner matchen efter ytterligare två vunna bollar. 
Poängen nollställs, och Henry vinner därefter en till boll. Därför står det \texttt{0-1}.